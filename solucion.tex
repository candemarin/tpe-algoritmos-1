\documentclass[a4paper]{article}

\setlength{\parskip}{2mm}
\newcommand{\tab}{~ \qquad}
\input{Algo1Macros}
\usepackage{caratula} % Version modificada para usar las macros de algo1 de ~> https://github.com/bcardiff/dc-tex

\begin{document}

\titulo{TP de Especificación}
\subtitulo{BUSCAMINAS}
\fecha{30 de Marzo de 2022}
\materia{Algoritmos y Estructuras de Datos I}
\grupo{Grupo 11}

% Tipos de ejemplo
\newcommand{\dato}{\textit{Dato}}
\newcommand{\individuo}{\textit{Individuo}}

% Tipos de este tp
\newcommand{\pos}{\textit{pos}}
\newcommand{\tablero}{\textit{tablero}}
\newcommand{\jugadas}{\textit{jugadas}}
\newcommand{\banderitas}{\textit{banderitas}}

% Pongan cuantos integrantes quieran
\integrante{Marin, Candela Emilia}{1405/21}{canmarin17@gmail.com}
\integrante{Borja, Kurt}{695/19}{kuurtb@gmail.com}
\integrante{Figueroa Isarrualde, Leandro}{213/17}{leanisarrualde@gmail.com}
\integrante{Sánchez, Warren}{004/15}{tienelasrespuestas@les.luthier.com}

\maketitle


\section{Definición de Tipos}
\begin{description}
	\item type \pos = $\mathbb{Z} \times \mathbb{Z}$
	\item type \tablero = $\TLista{\TLista{\bool}}$
	\item type \jugadas = $\TLista{\pos \times \mathbb{Z}}$
	\item type \banderitas = $\TLista{\pos}$
\end{description}

\section{Constantes}
\pred{posValida}{t: \tablero, p: \pos}{
	\text{/* false si t es vacía */} \\\\
	0 \leq p[0] < |t| \yLuego 0 \leq p[1] \leq |t[1]|
}

\pred{sonAdyacentes}{t: \tablero, p: \pos, q: \pos}{
	\text{/* si p = q el predicado es true */} \\ \\
	\text{posValida(t, p)} \land \\ 
	\text{posValida(t, q)} \land \\ 
	((p[0] - q[0])^2 \leq 1 \land (p[1] - q[1])^2 \leq 1 )
}

\pred{posEnJugadas}{p: \pos, j: \jugadas}{
	(\exists i : \ent)(0 \leq i < |j| \yLuego j[i]_0 = pos)
}

\pred{juegoGanado}{t: \tablero, j: \jugadas}{
	juegoValido(t,j) \yLuego \\
	(\forall i : \ent)(0 \leq i < |j| \implicaLuego t[j[i]_{0_0}][j[i]_{0_1}] = false) \y |j| = \text{posSinMinas}(t)
}

\pred{juegoPerdido}{t: \tablero, j: \jugadas}{
	juegoValido(t,j) \yLuego \\
	(\exists i : \ent)(0 \leq i \leq |j| \yLuego t[j[i]_{0_0}][j[i]_{0_1}] = true)
}

\aux{posSinMinas}{t: \tablero}{$\mathbb{Z}$}{\sum_{i=0}^{|t|-1}  \sum_{j=0}^{|t[i]|-1} \IfThenElse{t[i][j] = false } {1} {0}}

\section{Problemas}
\subsection{Ejercicio 1}

\aux{minasAdyacentes}{t: \tablero, p: \pos}{$\mathbb{Z}$}{\sum_{i=0}^{|t|-1} \sum_{j=0}^{|t[i]|-1} \IfThenElse{\text{sonAdyacentes}(t, p, (i,j)) \land t[i][j]}{1}{0}}

\subsection{Ejercicio 2}

\pred{juegoValido}{t: \tablero, j: \jugadas}{
(\forall i : \mathbb{Z}) (0 \leq i < |j| \implicaLuego \text{posValida}(t,j[i]_0) \y \text{\#apariciones(j, j[i]) = 1}) \y \\ \text{minasAdyacentes}(t, j[i]_0) = j[i]_1)
}

\subsection{Ejercicio 3}

\begin{proc}{PlantarBanderita}{\In t: \tablero, \In j: \jugadas, \In p: \pos, \Inout b : \banderitas }{}
    \pre{ b = b_{0} \y posValida(t, p) \y \neg posEnJugadas(p,t)\y p \notin b_{0}}
    \post {b = \text{addFirst}(p, b_{0})}
\end{proc}

\subsection{Ejercicio 4}
\begin{proc}{perdió}{\In t: \tablero, \In j: \jugadas, \Out res: \bool}{}

\pre{\text{juegoValido}(t,j)}
\post{res = true \longleftrightarrow juegoPerdidio(t,j)}
	
\end{proc}

\subsection{Ejercicio 5}
\begin{proc}{ganó}{\In t: \tablero, \In j: \jugadas, \Out res: \bool}{}
	\pre{\text{juegoValido}
	(t,j)}
	
	\post{res = true \longleftrightarrow juegoGanado(t,j)}
\end{proc}

\subsection{Ejercicio 6}
\begin{proc}{jugar}{\In t: \tablero, \In b: \banderitas, \In p : \pos, \Inout j: \jugadas}{}
\pre{\text j = j_{0} \y {juegoValido}(t,j_{0}) \y \neg juegoGanado(t,j) \y \neg juegoPerdido(t,j) \y \neg posEnJugadas(j_0) \y p \notin b}
\post{j = addFirst((p, minasAdyacentes(t,p)), j_{0})}
\end{proc}

\subsection{Ejercicio 7}

\pred{caminoLibre}{t: \tablero, p0: \pos, p1: \pos}{
posValida(t,p0) \y posValida(t,p1) \yLuego  posYadyacentesSinMinas(t, p0) \y \\ (t[p1_{0}][p1_{1}] = false \y posYadyacentesSinMinas(t, p1) = false)
}

\pred{posYadyacentesSinMinas}{t: \tablero, p: \pos}{(\forall i : \mathbb{Z}) (p_{0} - 2 \leq i \leq p_{0} + 2) \y (\forall j : \mathbb{Z}) (p_{1} - 2 \leq j \leq p_{1} + 2) \implicaLuego \neg \exists t[i][j] = true}




\subsection{Ejercicio 8}

\begin{proc}{jugarPlus}{\In t: \tablero, \In b: \banderitas, \In p : \pos, \Inout j: \jugadas}{}
\pre{\tex j=j_{0}\y{juegoValido}(t:\tablero,p:\pos) \y p \notin j \y p \notin b}
\post{(\forall p_{1}:\pos)({juegoValido}(t:\tablero,p_{1}:\pos)\implicaLuego {caminoLibre}(t:\tablero,p:\pos,p_{1}:\pos)\y (\forall Pnuevo,j_{k}:\pos)({posicionValida}(t:\tablero,Pnuevo:\pos)\y(j_{k}\times minasAdyacentes(t,j_{k})\in j)\y(Pnuevo\times minasAdyacentes(t,Pnuevo) \notin j)\y(nuevaPos(t:\tablero,j_{k}:\pos,Pnuevo:\pos)==Pnuevo)\implicaLuego j=j_{0} ++ p_{1}\times minasAdyacentes(t,p_{1})++Pnuevo\times minasAdyacentes(t,Pnuevo)))}


 \aux{nuevaPos}{t:\tablero,p:\pos,Pnuevo:\pos}{\pos}{\IfThenElse{\text sonAdyacentes(p:\pos,Pnuevo:\pos)\y minasAdyacentes(t:\tablero,Pnuevo:\pos)==0} {Pnuevo} {p}}
 
\end{proc}

\subsection{Ejercicio 9}

\begin{proc}{sugerirAutomatico121}{\In t: \tablero, \In b: \banderitas, \In j: jugadas, \Out p: \pos}{}

	\pre{juegoValido(t,j) \y banderitasValidas(t, b) \y tiene121Valido(t,b,j)}
	\post{posValida(t, p) \y \neg posEnJugadas(p,j) \y p \notin b \y adyacente121Valido(t,b,j,p)}
	
	\pred{tiene121Valido}{t: \tablero, b: \banderitas, j: \jugadas}{
		\text{/* jugadas tiene 1-2-1 y alguna pos libre que no está en b ni j*/} \\ \\ (\exists j1, j2, j3: \pos \times \ent) \\
		(j1,j2,j3 \in j \y \\ esSecuencia121(t, j1,j2,j3) \y \\ (\exists p': \pos)(sweetSpot(t, j1,j2,j3,p') \y \neg posEnJugadas(p',j) \y p' \notin b) )
	}
	
	\pred{adyacente121Valido}{t: \tablero, b: \banderitas, j: \jugadas, p: \pos}{
		\text{/* jugadas tiene 1-2-1 y p es adyacente y libre */} \\ \\
		(\exists j1, j2, j3: \pos \times \ent) \\
		(j1,j2,j3 \in j \y \\ esSecuencia121(t, j1,j2,j3) \y \\ sweetSpot(t,j1_0,j2_0,j3_0, p))
	}
	
	\pred{esSecuencia121}{t: \tablero, j1: \pos \ x \ent, j2: \pos \ x \ent, j3: \pos \ x \ent}{
		juegoValido(t, <j1,j2,j3>) \y \text{\hspace{1cm} /* verifica todos distintos en rango */} \\ j1_1 = 1 \y j2_1 = 2 \y j3_1 = 1 \y \\ estanEnLineaReacta(t, j1_0,j2_0,j3_0)
	}
	
	\pred{sweetSpot}{t: \tablero, p1: \pos, p2: \pos, p3: \pos, q: \pos}{
		\text{/* true si q está en el sweet spot */} \\
		\\ (sonAdyacentes(t,p1,q) \y sonAdyacentes(t,p2,q) \y sonAdyacentes(t,p3,q)) \lor \\ (sonAdyacentes(t,p1,q) \y \neg sonAdyacentes(t,p2,q) \y \neg sonAdyacentes(t,p3,q)) \lor \\ (\neg sonAdyacentes(t,p1,q) \y \neg sonAdyacentes(t,p2,q) \y sonAdyacentes(t,p3,q)) 
	}
	
	\pred{estanEnLineaRecta}{t: \tablero, p1: \pos, p2: \pos, p3: \pos}{
		sonAdyacentes(t,p1,p2) \y sonAdyacentes(t,p2,p3) \y \neg sonAdyacentes(t,p1,p3)
	}
	
	\pred{banderitasValidas}{t: \tablero, b: \banderitas}{
		(\forall i: \ent)(0 \leq i < |b| \implicaLuego posValida(t, b[i]))
	}
\end{proc}


% REFERENCIAS - NO BORRAR HASTA TERMINAR EL TP

\section{Ejemplos}
\subsection{Ejercicio 1}

\begin{proc}{unProblema}{\In t: \dato, \Inout result: $\bool$}{}
    \pre{\True}
    \post{result = \True \leftrightarrow predicadoPrincipal(t)}
    
    \pred{predicadoPrincipal}{d:\dato}{
    	pred1(d) \y pred2(d)
    }
    
    \pred{pred1}{d:\dato} {...}
    \pred{pred2}{d:\dato} {...}
\end{proc}




\subsection{Ejercicio 2}
\begin{proc}{otroProblema}{\In i: \individuo, \Inout result: $\bool$}{}
	\pre{|i| > MIN()}
	\post{result = \True \leftrightarrow predicadoPrincipal2(i)}
	
	\pred{predicadoPrincipal2}{i:\individuo}{ 
		pred1(i) \y \neg pred3(i)
	}
	
	\pred{pred3}{i:\individuo} {...}
\end{proc}



\end{document}
