\documentclass[a4paper]{article}

\setlength{\parskip}{2mm}
\newcommand{\tab}{~ \qquad}
\input{Algo1Macros}
\usepackage{caratula} % Version modificada para usar las macros de algo1 de ~> https://github.com/bcardiff/dc-tex

\begin{document}

\titulo{TP de Especificación}
\subtitulo{BUSCAMINAS}
\fecha{30 de Marzo de 2022}
\materia{Algoritmos y Estructuras de Datos I}
\grupo{Grupo 11}

% Tipos de ejemplo
\newcommand{\dato}{\textit{Dato}}
\newcommand{\individuo}{\textit{Individuo}}

% Tipos de este tp
\newcommand{\pos}{\textit{pos}}
\newcommand{\tablero}{\textit{tablero}}
\newcommand{\jugadas}{\textit{jugadas}}
\newcommand{\banderitas}{\textit{banderitas}}

% Pongan cuantos integrantes quieran
\integrante{Marin, Candela Emilia}{1405/21}{canmarin17@gmail.com}
\integrante{Borja, Kurt}{695/19}{kuurtb@gmail.com}
\integrante{Figueroa Isarrualde, Leandro}{213/17}{leanisarrualde@gmail.com}
\integrante{Messer, Santiago}{455/18}{santimesser@gmail.com}

\maketitle


\section{Definición de Tipos}
\begin{description}
	\item type \pos = $\mathbb{Z} \times \mathbb{Z}$
	\item type \tablero = $\TLista{\TLista{\bool}}$
	\item type \jugadas = $\TLista{\pos \times \mathbb{Z}}$
	\item type \banderitas = $\TLista{\pos}$
\end{description}

\section{Constantes}
\pred{posValida}{t: \tablero, p: \pos}{
	\text{/* false si t es vacía */} \\\\
	0 \leq p_0 < |t| \yLuego 0 \leq p_1 < |t[1]|
}

\pred{tableroValido}{t: \tablero}{
	(\forall i : \ent)(0 \leq i < |t| \implicaLuego |t[i]| = |t|) 
}

\pred{sonAdyacentes}{t: \tablero, p: \pos, q: \pos}{
	\text{/* si p = q el predicado es true */} \\ \\
	\text{posValida(t, p)} \land \\ 
	\text{posValida(t, q)} \land \\ 
	((p_0 - q_0)^2 \leq 1 \land (p_1 - q_1)^2 \leq 1 )
}

\pred{posEnJugadas}{p: \pos, j: \jugadas}{
	(\exists i : \ent)(0 \leq i < |j| \yLuego j[i]_0 = p)
}

\pred{juegoGanado}{t: \tablero, j: \jugadas}{
	juegoValido(t,j) \yLuego \\
	(\forall i : \ent)(0 \leq i < |j| \implicaLuego t[j[i]_{0_0}][j[i]_{0_1}] = false) \y |j| = \text{\#posSinMinas}(t)
}

\pred{juegoPerdido}{t: \tablero, j: \jugadas}{
	juegoValido(t,j) \yLuego \\
	(\exists i : \ent)(0 \leq i < |j| \yLuego t[j[i]_{0_0}][j[i]_{0_1}] = true)
}

\aux{\#posSinMinas}{t: \tablero}{$\mathbb{Z}$}{\sum_{i=0}^{|t|-1}  \sum_{j=0}^{|t[i]|-1} \IfThenElse{t[i][j] = false } {1} {0}}

\newpage

\section{Problemas}
\subsection{Ejercicio 1}

\aux{minasAdyacentes}{t: \tablero, p: \pos}{$\mathbb{Z}$}{\sum_{i=0}^{|t|-1} \sum_{j=0}^{|t[i]|-1} \IfThenElse{\text{sonAdyacentes}(t, p, (i,j)) \land t[i][j] = true}{1}{0}}

\subsection{Ejercicio 2}

\pred{juegoValido}{t: \tablero, j: \jugadas}{
tableroValido(t) \y \\
(\forall i : \mathbb{Z}) (0 \leq i < |j| \implicaLuego \text{posValida}(t,j[i]_0) \y \text{\#apariciones(j, j[i]) = 1}) \y \\ \text{minasAdyacentes}(t, j[i]_0) = j[i]_1)
}

\subsection{Ejercicio 3}

\begin{proc}{PlantarBanderita}{\In t: \tablero, \In j: \jugadas, \In p: \pos, \Inout b : \banderitas }{}
    \pre{ b = b_{0} \y juegoValido(t, j) \y posValida(t, p) \y \neg posEnJugadas(p,t)\y p \notin b_{0}}
    \post {b = \text{addFirst}(p, b_{0})}
\end{proc}

\subsection{Ejercicio 4}
\begin{proc}{perdió}{\In t: \tablero, \In j: \jugadas, \Out res: \bool}{}

\pre{\text{juegoValido}(t,j)}
\post{res = true \longleftrightarrow juegoPerdidio(t,j)}
	
\end{proc}

\subsection{Ejercicio 5}
\begin{proc}{ganó}{\In t: \tablero, \In j: \jugadas, \Out res: \bool}{}
	\pre{\text{juegoValido}
	(t,j)}
	
	\post{res = true \longleftrightarrow juegoGanado(t,j)}
\end{proc}

\subsection{Ejercicio 6}
\begin{proc}{jugar}{\In t: \tablero, \In b: \banderitas, \In p : \pos, \Inout j: \jugadas}{}
\pre{\text j = j_{0} \y {juegoValido}(t,j_{0}) \y \neg juegoGanado(t,j) \y \neg juegoPerdido(t,j) \y \neg posEnJugadas(j_0) \y p \notin b}
\post{j = addFirst((p, minasAdyacentes(t,p)), j_{0})}
\end{proc}

\subsection{Ejercicio 7}

\pred{caminoLibre}{t: \tablero, p0: \pos, p1: \pos}{
	posValida(p0) \y posValida(p1) \y \\ 
	minasAdyacentes(t, p0) = 0 \y \\ 
	minasAdyacentes(t, p1) \geq 1 \y t[p1_0][p1_1] = false \y \\ \\\text{/* Existe un camino de p1 a p0 valido */} \\
	(\exists s: \TLista{\pos})(|s| > 0 \yLuego s[0] = p1 \y s[|s| - 1] = p0 \y seqEsCaminoLibre(t, s))
}

\pred{seqEsCaminoLibre}{t: \tablero, s: \TLista{\pos}}{
	sinRepetidos(s) \y seqPosValidas(t, s) \y esCamino(s) \y seqSinBombas(\text{tail}(s))
}

\pred{sinRepetidos}{s: \TLista{\pos}}{
	(\forall i: \ent)(0 \leq i < |s| \implicaLuego \text{\#apariciones}(s[i]) = 1)
}

\pred{seqPosValidas}{t: \tablero, s: \TLista{\pos}}{
	(\forall i: \ent)(0 \leq i < |s| \implicaLuego posValida(t, s[i]))
}

\pred{esCamino}{s: \TLista{\pos}}{
	(\forall i: \ent)(1 \leq i < |s| \implicaLuego aUnPaso(s[i-1], s[i]))
}

\pred{aUnPaso}{p: \pos, q: \pos}{
	(p_0 - q_0)^2 + (p_1-q_1)^2 = 1
}

\pred{seqSinBombas}{t: \tablero, s: \TLista{\pos}}{
	(\forall i: \ent)(0 \leq i < |s| \implicaLuego minasAdyacentes(t, s[i]) = 0)
}


\subsection{Ejercicio 8}
\begin{proc}{jugarPlus}{\In t: \tablero, \In b: \banderitas, \In p : \pos, \Inout j: \jugadas}{}
	\pre{\text j = j_{0} \y {juegoValido}(t,j_{0}) \y \neg juegoGanado(t,j) \y \neg juegoPerdido(t,j) \y \neg posEnJugadas(j_0) \y p \notin b}
	
	\post{posEnJugadas(p, j) \y estaContenida(j_0, j) \y sinRepetidos(j) \y \\ (\forall x: \pos \times \ent)(esNuevaJugada(x) \longleftrightarrow caminoLibreV2(t, p, x))}
	
	\pred{estaContenida}{s: \TLista{T}, t: \TLista{T}}{
		(\forall x: T)(x \in s \implica x \in t)
	}
	
	\pred{esNuevaJugada}{x: \pos \, x \ent}{
		\text{\#apariciones}(j_0, x) = 0 \y \\ \text{\#apariciones}(j, x) = 1
	}
	
	\pred{caminoLibreV2}{t: \tablero, p0: \pos, p1: \pos}{
	\text{/* Igual a caminoLibre pero minasAdyacentes(p1) puede ser 0 */} \\\\
	posValida(p0) \y posValida(p1) \y \\ 
	minasAdyacentes(t, p0) = 0 \y \\ 
	t[p1_0][p1_1] = false \y \\ \\\text{/* Existe un camino de p1 a p0 valido */} \\
	(\exists s: \TLista{\pos})(|s| > 0 \yLuego s[0] = p1 \y s[|s| - 1] = p0 \y seqEsCaminoLibre(t, s))
	}
	
\end{proc}


\subsection{Ejercicio 9}

\begin{proc}{sugerirAutomatico121}{\In t: \tablero, \In b: \banderitas, \In j: jugadas, \Out p: \pos}{}

	\pre{juegoValido(t,j) \y banderitasValidas(t, b) \y tiene121Valido(t,b,j)}
	\post{posValida(t, p) \y \neg posEnJugadas(p,j) \y p \notin b \y adyacente121Valido(t,b,j,p)}
	
	\pred{tiene121Valido}{t: \tablero, b: \banderitas, j: \jugadas}{
		\text{/* jugadas tiene 1-2-1 y alguna pos libre que no está en b ni j*/} \\ \\ (\exists j1, j2, j3: \pos \times \ent) \\
		(j1,j2,j3 \in j \y \\ esSecuencia121(t, j1,j2,j3) \y \\ (\exists p': \pos)(estaEnPosSinMina(t, j1,j2,j3,p') \y \neg posEnJugadas(p',j) \y p' \notin b) )
	}
	
	\pred{adyacente121Valido}{t: \tablero, b: \banderitas, j: \jugadas, p: \pos}{
		\text{/* jugadas tiene 1-2-1 y p es adyacente y libre */} \\ \\
		(\exists j1, j2, j3: \pos \times \ent) \\
		(j1,j2,j3 \in j \y \\ esSecuencia121(t, j1,j2,j3) \y \\ estaEnPosSinMina(t,j1_0,j2_0,j3_0, p))
	}
	
	\pred{esSecuencia121}{t: \tablero, j1: \pos \ x \ent, j2: \pos \ x \ent, j3: \pos \ x \ent}{
		juegoValido(t, <j1,j2,j3>) \y \text{\hspace{1cm} /* verifica todos distintos en rango */} \\ j1_1 = 1 \y j2_1 = 2 \y j3_1 = 1 \y \\ estanEnLineaReacta(t, j1_0,j2_0,j3_0)
	}
	
	\pred{estaEnPosSinMina}{t: \tablero, p1: \pos, p2: \pos, p3: \pos, q: \pos}{
		\text{/* true si q está una pos sin mina */} \\
		\\ (sonAdyacentes(t,p1,q) \y sonAdyacentes(t,p2,q) \y sonAdyacentes(t,p3,q)) \lor \\ (sonAdyacentes(t,p1,q) \y \neg sonAdyacentes(t,p2,q) \y \neg sonAdyacentes(t,p3,q)) \lor \\ (\neg sonAdyacentes(t,p1,q) \y \neg sonAdyacentes(t,p2,q) \y sonAdyacentes(t,p3,q)) 
	}
	
	\pred{estanEnLineaRecta}{t: \tablero, p1: \pos, p2: \pos, p3: \pos}{
		aUnPaso(t,p1,p2) \y aUnPaso(t,p2,p3) \y \neg sonAdyacentes(t,p1,p3)
	}
	
	\pred{banderitasValidas}{t: \tablero, b: \banderitas}{
		(\forall i: \ent)(0 \leq i < |b| \implicaLuego posValida(t, b[i]))
	}
\end{proc}


\end{document}
